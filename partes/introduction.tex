\label{sec:introducao}

A introdução tem como objetivo contextualizar o tema do trabalho, apresentando ao leitor uma visão geral do problema que será investigado. É nesta seção que se constrói a justificativa para a realização do estudo, indicando sua relevância acadêmica, científica, social ou prática.

De forma geral, a introdução deve abordar os seguintes elementos:

\begin{itemize}
    \item \textbf{Contextualização}: apresentar o tema, situando-o no contexto social, científico ou tecnológico, de forma a demonstrar sua importância.
    
    \item \textbf{Delimitação do problema}: definir claramente qual é o problema que será investigado, especificando os limites e o foco do estudo.
    
    \item \textbf{Justificativa}: argumentar sobre a relevância do trabalho, destacando sua contribuição teórica, prática, social ou científica.
    
    \item \textbf{Objetivos}: apresentar o \textbf{objetivo geral} e, quando aplicável, os \textbf{objetivos específicos}, que indicam os passos necessários para alcançar o objetivo principal.
    
    \item \textbf{Estrutura do trabalho}: descrever brevemente como o trabalho está organizado, indicando o conteúdo de cada capítulo.
\end{itemize}

A introdução deve ser escrita de forma clara, objetiva e coerente, conduzindo o leitor naturalmente para a compreensão da proposta do trabalho e da sua importância no contexto da área de estudo.

\subsecao{Citações e Referências}
\label{subsec:citref}

Durante a elaboração de trabalhos acadêmicos, é fundamental realizar citações corretas, reconhecendo as ideias de outros autores que servem de base para a construção do conhecimento. No ambiente \LaTeX, o gerenciamento de referências é feito, preferencialmente, por meio de um arquivo auxiliar com extensão \texttt{.bib}, onde são cadastradas todas as fontes bibliográficas utilizadas.

\subsubsecao{Citações no Texto}

As citações podem ser diretas (transcrição literal) ou indiretas (paráfrase). No \LaTeX, a citação no texto é feita utilizando comandos específicos, como:

\begin{itemize}
    \item \texttt{\textbackslash cite\{chave\}} — gera uma citação simples no formato autor-data ou numérico, conforme o estilo definido.
    \item \texttt{\textbackslash citeonline\{chave\}} — (em alguns pacotes) cita o autor no texto seguido do ano entre parênteses.
\end{itemize}

Exemplo no texto:

\begin{verbatim}
Segundo \cite{sobrenomeAno}, a metodologia é essencial para...
\end{verbatim}

Que gera uma citação como:

\textit{Segundo Sobrenome (Ano), a metodologia é essencial para...}

Um exemplo prático é: Segundo \cite{Knuth_1986_texbook}, ...

\subsubsecao{Referências}
\label{subsubsec:referencias}

É comum uma seção ou subseção referenciar outra subseção. Para isso, deve-se utilizar o comando \verb+\label+ para gerar rótulos que poderão ser referenciados no restante do documento. Isso pode ser feito da seguinte forma:

\begin{verbatim}
\label{sec:introducao}
\end{verbatim}

Ao inserir este no início do texto, a qualquer momento do texto, pode-se escrever o seguinte:
\begin{verbatim}
Esta é uma referência à seção~\ref{sec:introducao}.
\end{verbatim}

Como exemplo, a Subseção~\ref{subsec:citref} explica sobre Citações e Referências.

Referências a apêndices é um caso especial, pois essa referência utiliza letras maiúsculas, ao invés de um contador numérico. Por causa disso, a criação de rótulos é feita direta no comando de inserção do apêndice (ao invés do uso de \verb+\label+). Ao inserir um apêndice, pode-se utilizar uma das seguintes opções:

\begin{enumerate}
    \item \verb+\apendice[ap:meuapendice]{Primeiro apêndice}{partes/apendicei}+
    \item \verb+\apendice{Segundo apêndice}{partes/apendiceii}+
\end{enumerate}

Repare que apenas no primeiro caso o parâmetro que define o label para o Apêndice~\ref{ap:meuapendice} foi informado, e será aquele que poderá ser utilizado com o comando \verb+\ref+.

